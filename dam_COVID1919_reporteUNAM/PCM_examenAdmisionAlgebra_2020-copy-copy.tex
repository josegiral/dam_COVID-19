%\input{tcilatex}


%\documentclass{article}
\documentclass[addpoints,answers,10pt]{exam}
%%%%%%%%%%%%%%%%%%%%%%%%%%%%%%%%%%%%%%%%%%%%%%%%%%%%%%%%%%%%%%%%%%%%%%%%%%%%%%%%%%%%%%%%%%%%%%%%%%%%%%%%%%%%%%%%%%%%%%%%%%%%
\usepackage[utf8]{inputenc}
\usepackage[spanish,es-nodecimaldot]{babel}
%\usepackage{fancyhdr}
\usepackage{graphicx}
\usepackage{amsfonts,amsmath,amssymb}
%\usepackage{fullpage}
\usepackage{geometry}
\usepackage{tikz}
\usetikzlibrary{positioning, arrows.meta}
\newcommand{\here}[2]{\tikz[remember picture]{\node[inner sep=0](#2){#1}}}

\setcounter{MaxMatrixCols}{10}
%TCIDATA{OutputFilter=LATEX.DLL}
%TCIDATA{Version=4.00.0.2312}
%TCIDATA{Created=Sat Jun 01 15:05:36 2002}
%TCIDATA{LastRevised=Friday, October 02, 2009 13:05:58}
%TCIDATA{<META NAME="GraphicsSave" CONTENT="32">}
%TCIDATA{<META NAME="DocumentShell" CONTENT="General\Blank Document">}
%TCIDATA{CSTFile=LaTeX article (bright).cst}
%This is where you can print out your solutions

%\printanswers
\noprintanswers

%\newtheorem{theorem}{Theorem}
%\newtheorem{acknowledgement}[theorem]{Acknowledgement}
%\newtheorem{algorithm}[theorem]{Algorithm}
%\newtheorem{axiom}[theorem]{Axiom}
%\newtheorem{case}[theorem]{Case}
%\newtheorem{claim}[theorem]{Claim}
%\newtheorem{conclusion}[theorem]{Conclusion}
%\newtheorem{condition}[theorem]{Condition}
%\newtheorem{conjecture}[theorem]{Conjecture}
%\newtheorem{corollary}[theorem]{Corollary}
%\newtheorem{criterion}[theorem]{Criterion}
%\newtheorem{definition}[theorem]{Definition}
%\newtheorem{example}[theorem]{Example}
%\newtheorem{exercise}[theorem]{Exercise}
%\newtheorem{lemma}[theorem]{Lemma}
%\newtheorem{notation}[theorem]{Notation}
%\newtheorem{problem}[theorem]{Problem}
%\newtheorem{proposition}[theorem]{Proposition}
%\newtheorem{remark}[theorem]{Remark}
%\newtheorem{solution}[theorem]{Solution}
%\newtheorem{summary}[theorem]{Summary}
\newenvironment{proof}[1][Proof]{\textbf{#1.} }{\ \rule{0.5em}{0.5em}}
%
\providecommand{\lrSet}[1]{\left\{#1\right\}}
\providecommand{\lrRound}[1]{\left(#1\right)}
\providecommand{\lrSquare}[1]{\left[#1\right]}
\providecommand{\abs}[1]{\lvert#1\rvert}
\providecommand{\norm}[1]{\lVert#1\rVert}
%\input{tcilatex}
%\pagestyle{fancy}
\geometry{left=2cm,right=2cm,top=2cm,bottom=1cm}
%\setlength\headwidth{\textwidth}
%\renewcommand{\footrulewidth}{0pt}
%\renewcommand{\headrulewidth}{2pt}
% ----------------------------------------------------
\begin{document}
\pointpoints{punto}{puntos}
\vqword{Problema}
\vpword{Max}
\vsword{Puntos}

\lhead{\textbf{Posgrado en Ciencias Matemáticas, UNAM\\Examen de Admisi\'on de \'Algebra Lineal, 2020}}
\rhead{Código \rule{3cm}{1pt}}

\bigskip
\begin{minipage}{0.6\textwidth}
\paragraph{Instrucciones.} Escriba su nombre y un código de 2 números y 2 letras, todos distintos, en la lista de asistentes a este examen. Escriba una hoja de respuestas y entréguela junto con el resto de su trabajo escribiendo su código en cada página. Tiene un máximo de dos horas para entregar sus respuestas. Comience a escribir las respuestas de su examen al menos 15 minutos antes de la hora límite. Sólo las respuestas contenidas en la hoja de respuestas serán consideradas para la calificación. 
Favor de no escribir en la tabla de la derecha.
\end{minipage}%
\begin{minipage}{0.1\textwidth}
\hfill
\end{minipage}%
\begin{minipage}{0.3\textwidth}
\gradetablestretch{1.1}
\addpoints	
\gradetable[v][questions]
\end{minipage}

\paragraph{Preguntas}
% ------------------
\begin{questions}
% ------------------

% ------------------
\question[10] 
Si $A=\left( 
\begin{array}{rr}
-5 & 18 \\ 
-6 & 7 %
\end{array}
\right)$ encuentre una expresión generalizada para $A^n$, para $n \in \mathbb{N}$. 

% ------------------
%\question 
%\begin{parts} 
%\part[10] Sea $A \in M_{n}\lrRound{\mathbb{K}}$ no invertible.  Demuestre que existe $B\in M_n\lrRound{\mathbb{K}}$  no nula  tal que $AB=0.$
%\part[10] Si $A=\left( 
%\begin{array}{rrr}
%1 & 4 & 7\\ 
%2 & 1 & 2\\
%4 & 3 & 5\\
%\end{array}
%\right)$, encuentre $B \in M_{3}\lrRound{\mathbb{R}}$   tal que $AB=0.$
%\end{parts}


% ------------------
\question Sea $T: \mathbb{R}^{2} \rightarrow \mathbb{R}^{2}: u \mapsto Au$, con $A$ simétrica. 
\begin{parts}
\part[10] ?`Son perpendiculares los vectores propios de $A$? Argumente formalmente su respuesta.
\part[10] Calcule los valores y vectores propios de $A=\left( 
\begin{array}{rr}
1 & 1 \\ 
1& -5 %
\end{array}
\right).$
\part[10] Explique qué representan los valores propios de $A$ e ilustre con un dibujo. Pista: considere las imágenes de los puntos $\lrSet{(1,0), (0,1), (-1,0), (0,-1)}$ bajo $T$.  \\
\end{parts} 


% ------------------
\question Sean $b \in \mathbb{R}^{n}$ y $A=\lrSquare{a_{1} \quad ... \quad a_{n}} \in M_{n}\lrRound{\mathbb{R}}$. Defina 
\[
A_i(b) =(a_{1} \ldots \here{b}{fromhere}\ldots a_{n})
\]
\begin{tikzpicture}[remember picture, overlay]
\node[font=\scriptsize, below right=12pt of fromhere] (tohere) {$i$-ésima entrada};
\draw[Stealth-] ([yshift=-4pt]fromhere.south) |- (tohere);
\end{tikzpicture}

como la matriz $A$ con la $i$-ésima columna reemplazada por el vector $b$. 
\begin{parts}
\part[5] Pruebe que $A~I_{i}(x)=A_{i}(b)$ para todo $x \in \mathbb{R}^{n}$. 
\part[10] Usando el inciso anterior, demuestre la regla de Cramer: Si $A$ es invertible, la única solución de $Ax=b$ tiene entradas de la forma
$$x_{i} = \frac{\det A_{i}(b)}{\det A}.$$ 
Pista: $\det I_{i}(x) = x_{i}$.
\end{parts}

% ------------------
\question Sea $\mathcal{B}=\lrSet{\mathbf{b_{1}},\ldots,\mathbf{b_{n}}}$ una base ordenada para un espacio vectorial $V$ sobre $\mathbb{R}$. 
\begin{parts}
\part[10] Demuestre que el mapeo de coordenadas 
$\mathbf{x} \mapsto \lrSquare{ \mathbf{x}}_{\mathcal{B}}$ es una transformación lineal $V\rightarrow \mathbb{R}^n$. 
\part[10] Explique por qué el vector de $\mathcal{B}$-coordenadas de una combinación lineal de vectores $\mathbf{u_{1}},\ldots,\mathbf{u_{p}}~\in~V$ es la misma combinación lineal de los vectores de coordenadas $\lrSet{\lrSquare{\bf{u_{i}}}_{\mathcal{B}}:i=1,...,n}$
\end{parts}


% ------------------
\question Sea 
$$T: C(\mathbb{R},\mathbb{R}) \rightarrow C(\mathbb{R},\mathbb{R}) : f(x) \mapsto \frac{f(x) + f(-x)}{2}$$ 
\begin{parts}
\part[10] Demuestre que $T$ es lineal
\part[10] Encuentre el n\'{u}cleo y la imagen de $T$
\end{parts}

% ------------------
\question[20] Sea $V$ un espacio vectorial sobre $\mathbb{C}$,  con producto interno, y de dimensi\'on $n$. Sea $T$ un operador lineal en $V$. Demuestre que si $\{v_1,\dots ,v_n\}$ es una base ortonormal de $V$ y su imagen bajo $T$ es otra base ortonormal de $V$, entonces $\norm{T(v)}=\norm{v}$ para todo $v\in V$.



% ------------------
%\question Considere el espacio vectorial $V=M_{n}\lrRound{\mathbb{R}}$ sobre $\mathbb{R}$, y un operador $T:V\rightarrow V$ definido por $T(A)=A^t$. 
%\begin{parts}
%\part[5] 
%Pruebe que $1$ y $-1$ son los únicos valores propios de $T$.
%\part[5] 
%Describa los vectores propios correspondientes.
%\part[10] 
%Encuentre una base $\mathcal{B}$ tal que $[T]_{\mathcal{B}}$ sea diagonal.
%\end{parts}


% ------------------
%\question[10] Considere el subespacio $W$ de $\mathbb{R}^3$ definido como el generado por los vectores $v_1=(1,0,-1)$, ${v_2=(2,-3,0)}$, $v_3=(-5,1,-3)$.  Encuentre una transformaci\'on lineal  $T:\mathbb{R}^3\rightarrow \mathbb{R}^4$ tal que $W$ sea el n\'ucleo de $T$.

\end{questions}


% ------------------
\paragraph{Notación y definiciones.} 
$M_{m}\lrRound{\mathbb{K}}$ es el espacio de matrices de $m\times m$  con coeficientes en $\mathbb{K}$.   Dada una base $\mathcal{B}=\lrSet{\mathbf{b}_{i}:i=1,...,n}$ de un espacio vectorial $V$ y un vector $\mathbf{x}\in V$, las $\mathcal{B}$-coordenadas de x son los pesos $c_{1}$,...., $c_{n}$ tales que $\mathbf{x}= c_{1}\mathbf{b}_{1}+\ldots+c_{n}\mathbf{b}_{n}$. $C(\mathbb{R},\mathbb{R}) $ es el espacio de funciones contínuas de $\mathbb{R}$ en $\mathbb{R}$. 



\end{document}
